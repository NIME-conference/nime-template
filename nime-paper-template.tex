% Template for NIME 2025

% Modified by Florent Berthaut 31 October 2024
% Modified by Adnan Marquez-Borbon 30 November 2022
% Modified by Courtney Reed 28 November 2022
% Modified by Joe Wright 14 December 2019
% Modified by Niccolò Granieri 10 October 2018 
% Modified by Angelo Fraietta 23 December 2018
% Modified by Angelo Fraietta 22 November 2018
% Modified by Rodrigo Schramm on 22 September 2018
% Modified by Luke Dahl on 17 October 2-17
% Modified by Cumhur Erkut on <2016-10-11 Tue>
% Modified by Edgar Berdahl on 5 November 2014
% Modified by Baptiste Caramiaux on 25 November 2013
% Modified by Kyogu Lee on 7 October 2012
% Modified by Georg Essl on 7 November 2011
%
% Based on "sig-alternate.tex" V1.9 April 2009
% This file should be compiled with "nime-alternate.cls"

\documentclass{nime-alternate}

% Uncomment only one of the ones below
\usepackage{anonymize} 		   %Uncomment this line to publish
%\usepackage[blind]{anonymize}%Uncomment this line for blind review

% Package that enables the use of accents and non 
% standard characters
\usepackage[utf8]{inputenc}

% Package and option to handle bibliography
\usepackage[numbers]{natbib}
\bibliographystyle{ACM-Reference-Format}

\begin{document}

% --- Author Metadata here ---
%\conferenceinfo{NIME'17,}{May 15-19, 2017, Aalborg University Copenhagen, Denmark.}
%\conferenceinfo{NIME'18,}{June 3-6, 2018, Blacksburg, Virginia, USA.}
%\conferenceinfo{NIME'19,}{June 3-6, 2019, Federal University of Rio Grande do Sul, ~~~~~~  Porto Alegre,  Brazil.}
% \conferenceinfo{NIME'20,}{July 21-25, 2020, Royal Birmingham Conservatoire, ~~~~~~~~~~~~ Birmingham City University, Birmingham, United Kingdom.}
%\conferenceinfo{NIME'22,}{June 28 - July 1, 2022, Waipapa Taumata Rau, T\={a}maki ~~~~~~~~ Makaurau, Aotearoa}
%\conferenceinfo{NIME'24,}{4--6 September, Utrecht, The Netherlands.}

\conferenceinfo{Proceedings of the International Conference on New Interfaces for Musical Expression (NIME'25).}{June 24--27, 2025. The Australian National University, Canberra, Australia.}

\title{NIME Proceedings Template for LaTeX}

% You need the command \numberofauthors to handle the 'placement
% and alignment' of the authors beneath the title.
%
% For aesthetic reasons, we recommend 'three authors at a time'
% i.e. three 'name/affiliation blocks' be placed beneath the title.
%
% NOTE: You are NOT restricted in how many 'rows' of
% "name/affiliations" may appear. We just ask that you restrict
% the number of 'columns' to three.
%
% Because of the available 'opening page real-estate'
\label{key}
% we ask you to refrain from putting more than six authors
% (two rows with three columns) beneath the article title.
% More than six makes the first-page appear very cluttered indeed.
%
% Use the \alignauthor commands to handle the names
% and affiliations for an 'aesthetic maximum' of six authors.
% Add names, affiliations, addresses for
% the seventh etc. author(s) as the argument for the
% \additionalauthors command.
% These 'additional authors' will be output/set for you
% without further effort on your part as the last section in
% the body of your article BEFORE References or any Appendices.

\numberofauthors{6}

\author{
% You can go ahead and credit any number of authors here,
% e.g. one 'row of three' or two rows (consisting of one row of three
% and a second row of one, two or three).
%
% The command \alignauthor (no curly braces needed) should
% precede each author name, affiliation/snail-mail address and
% e-mail address. Additionally, tag each line of
% affiliation/address with \affaddr, and tag the
% e-mail address with \email.
%
% 1st. author
\alignauthor
\anonymize{Ben Trovato}\\
       \affaddr{\anonymize{Institute for Clarity in Documentation}}\\
       \affaddr{\anonymize{1932 Wallamaloo Lane}}\\
       \affaddr{\anonymize{Wallamaloo, New Zealand}}\\
       \email{\anonymize{trovato@corporation.com}}
% 2nd. author
\alignauthor
\anonymize{G.K.M. Tobin}\\
       \affaddr{\anonymize{Institute for Clarity in Documentation}}\\
       \affaddr{\anonymize{P.O. Box 1212}}\\
       \affaddr{\anonymize{Dublin, Ohio 43017-6221}}\\
       \email{\anonymize{webmaster@marysville-ohio.com}}
% 3rd. author
\alignauthor \anonymize{Lars Th{\o}rv{\"a}ld}\\
       \affaddr{\anonymize{The Th{\o}rv{\"a}ld Group}}\\
       \affaddr{\anonymize{1 Th{\o}rv{\"a}ld Circle}}\\
       \affaddr{\anonymize{Hekla, Iceland}}\\
       \email{l\anonymize{arst@affiliation.org}}
\and  % use '\and' if you need 'another row' of author names
% 4th. author
\alignauthor \anonymize{Lawrence P. Leipuner}\\
       \affaddr{\anonymize{Brookhaven Laboratories}}\\
       \affaddr{\anonymize{Brookhaven National Lab}}\\
       \affaddr{\anonymize{P.O. Box 5000}}\\
       \email{\anonymize{lleipuner@researchlabs.org}}
% 5th. author
\alignauthor \anonymize{Sean Fogarty}\\
       \affaddr{\anonymize{NASA Ames Research Center}}\\
       \affaddr{\anonymize{Moffett Field}}\\
       \affaddr{\anonymize{California 94035}}\\
       \email{\anonymize{fogartys@amesres.org}}
% 6th. author
\alignauthor \anonymize{Anon Nymous}\\
       \affaddr{\anonymize{Redacted }}\\
       \affaddr{\anonymize{8600 Datapoint Drive}}\\
       \affaddr{\anonymize{San Antonio, Texas 78229}}\\
       \email{\anonymize{cpalmer@prl.com}}
}

\maketitle

\begin{abstract}
This paper provides a sample of a \LaTeX\ document for the NIME conference series. It conforms,
somewhat loosely, to the formatting guidelines for
ACM SIG Proceedings. It is an {\em alternate} style which produces
a {\em tighter-looking} paper and was designed in response to
concerns expressed, by authors, over page-budgets.
It complements the document \textit{Author's (Alternate) Guide to
Preparing ACM SIG Proceedings Using \LaTeX$2_\epsilon$\ and Bib\TeX}.
This source file has been written with the intention of being
compiled under \LaTeX$2_\epsilon$\ and BibTeX.

To make best use of this sample document, run it through \LaTeX\
and BibTeX, and compare this source code with your compiled PDF file. A compiled PDF version is available to help you with the `look and feel.' 
\textbf{The paper submitted to the NIME conference must be stored in an \underline{A4}-sized PDF file, so North Americans should take care not to inadvertently generate \underline{letter} paper-sized PDF files.}  This paper template should prevent that from happening if the \texttt{pdflatex} program is used to generate the PDF file.

The abstract should preferably be between 100 and 200 words.
\end{abstract} 

\keywords{NIME, proceedings, \LaTeX, template}


\section{Introduction}
The \textit{proceedings} are the records of a conference.
ACM seeks to give these conference by-products a uniform,
high-quality appearance.  To do this, ACM has some rigid
requirements for the format of the proceedings documents: there
is a specified format (balanced  double columns), a specified
set of fonts (Arial or Helvetica and Times Roman) in
certain specified sizes (for instance, 9 point for body copy).

The good news is, with only a handful of manual
settings,\footnote{Two of these, the {\texttt{\char'134 numberofauthors}}
and {\texttt{\char'134 alignauthor}} commands, you have
already used; another, {\texttt{\char'134 balancecolumns}}, will
be used in your very last run of \LaTeX\ to ensure
balanced column heights on the last page.} the \LaTeX\ document
class file handles all of this for you.

The remainder of this document is concerned with showing, in
the context of an ``actual'' document, the \LaTeX\ commands
specifically available for denoting the structure of a
proceedings paper, rather than with giving rigorous descriptions
or explanations of such commands.

\section{The Body of The Paper}
Typically, the body of a paper is organized
into a hierarchical structure, with numbered or unnumbered
headings for sections, subsections, sub-subsections, and even
smaller sections.  The command \texttt{{\char'134}section} that
precedes this paragraph is part of such a
hierarchy.\footnote{This is the second footnote.  It
starts a series of three footnotes that add nothing
informational, but just give an idea of how footnotes work
and look. It is a wordy one, just so you see
how a longish one plays out.} \LaTeX\ handles the numbering
and placement of these headings for you, when you use
the appropriate heading commands around the titles
of the headings.  If you want a sub-subsection or
smaller part to be unnumbered in your output, simply append an
asterisk to the command name.  Examples of both
numbered and unnumbered headings will appear throughout the
balance of this sample document.

Because the entire article is contained in
the \textbf{document} environment, you can indicate the
start of a new paragraph with a blank line in your
input file; that is why this sentence forms a separate paragraph.

\subsection{Type Changes and Special Characters}

We have already seen several typeface changes in this sample.  You
can indicate italicized words or phrases in your text with
the command \texttt{{\char'134}textit}; emboldening with the
command \texttt{{\char'134}textbf}
and typewriter-style (for instance, for computer code) with
\texttt{{\char'134}texttt}.  But remember, you do not
have to indicate typestyle changes when such changes are
part of the \textit{structural} elements of your
article; for instance, the heading of this subsection will
be in a sans serif\footnote{A third footnote, here.
Let's make this a rather short one to
see how it looks.} typeface, but that is handled by the
document class file. Take care with the use
of\footnote{A fourth, and last, footnote.}
the curly braces in typeface changes; they mark
the beginning and end of
the text that is to be in the different typeface.

You can use whatever symbols, accented characters, or
non-English characters you need anywhere in your document;
you can find a complete list of what is
available in the \textit{\LaTeX\
User's Guide} \cite{Lamport:LaTeX}.

\subsection{Tables}
Because tables cannot be split across pages, the best
placement for them is typically the top of the page
nearest their initial cite.  To
ensure this proper ``floating'' placement of tables, use the
environment \textbf{table} to enclose the table's contents and
the table caption.  The contents of the table itself must go
in the \textbf{tabular} environment, to
be aligned properly in rows and columns, with the desired
horizontal and vertical rules.  Again, detailed instructions
on \textbf{tabular} material
is found in the \textit{\LaTeX\ User's Guide}.

Immediately following this sentence is the point at which
Table~\ref{tab:frequency} is included in the input file; compare the
placement of the table here with the table in the printed
dvi output of this document. 

To get correct numbering of tables, please use the \textbf{label} command inside the \textbf{table} environment and a corresponding \textbf{ref} in the text.

\begin{table}
\centering
\caption{Frequency of Special Characters}
\begin{tabular}{|c|c|l|} \hline
Non-English or Math&Frequency&Comments\\ \hline
\O & 1 in 1,000& For Swedish names\\ \hline
$\pi$ & 1 in 5& Common in math\\ \hline
\$ & 4 in 5 & Used in business\\ \hline
$\Psi^2_1$ & 1 in 40,000& Unexplained usage\\
\hline\end{tabular}
\label{tab:frequency}
\end{table}

To set a wider table, which takes up the whole width of
the page's live area, use the environment
\textbf{table*} to enclose the table's contents and
the table caption.  As with a single-column table, this wide
table will ``float" to a location deemed more desirable.
Immediately following this sentence is the point at which
Table 2 is included in the input file; again, it is
instructive to compare the placement of the
table here with the table in the printed dvi
output of this document.


\begin{table*}[htbp]
\centering
\caption{Some Typical Commands}
\begin{tabular}{|c|c|l|} \hline
Command&A Number&Comments\\ \hline
\texttt{{\char'134}alignauthor} & 100& Author alignment\\ \hline
\texttt{{\char'134}numberofauthors}& 200& Author enumeration\\ \hline
\texttt{{\char'134}table}& 300 & For tables\\ \hline
\texttt{{\char'134}table*}& 400& For wider tables\\ \hline\end{tabular}
\end{table*}
% end the environment with {table*}, NOTE not {table}!

\subsection{Figures}
Like tables, figures cannot be split across pages; the best placement for them is typically the top or the bottom of the page nearest their initial cite. To ensure this proper ``floating'' placement of figures, use the environment \textbf{figure} to enclose the figure and its caption. Optionally, for small figures that you want to place inline with the text, you can use the \textbf{htbp} options to \textbf{figure} to see whether there is space for placing it inline with the text.

This sample document contains examples of \textbf{.pdf} (Figure~\ref{fig:BlockDiagram1}) and \textbf{.jpg} files to be displayable with \LaTeX. More details on each of these is found in the \textit{Author's Guide}.

\begin{figure}[htbp]
	\centering
		\includegraphics[width=1\columnwidth]{BlockDiagram1}
	\caption{A sample image in PDF format.}
	\label{fig:BlockDiagram1}
\end{figure}

As was the case with tables, you may want a figure that spans two columns (Figure~~\ref{fig:BlockDiagram2}). To do this, and still to ensure proper ``floating'' placement of tables, use the environment \textbf{figure*} to enclose the figure and its caption. And don't forget to end the environment with {figure*}, not {figure}!

\begin{figure*}[htbp]
	\centering
		\includegraphics[width=1\textwidth]{BlockDiagram2}
	\caption{A sample black and white graphic (.jpg format) that needs to span two columns of text.}
	\label{fig:BlockDiagram2}
\end{figure*}

\subsection{Math Equations}
You may want to display math equations in three distinct styles:
inline, numbered or non-numbered display.  Each of
the three are discussed in the next sections.

\subsubsection{Inline (In-text) Equations}
A formula that appears in the running text is called an
inline or in-text formula.  It is produced by the
\textbf{math} environment, which can be
invoked with the usual \texttt{{\char'134}begin. . .{\char'134}end}
construction or with the short form \texttt{\$. . .\$}. You
can use any of the symbols and structures,
from $\alpha$ to $\omega$, available in
\LaTeX \cite{Lamport:LaTeX}; this section will simply show a
few examples of in-text equations in context. Notice how
this equation: \begin{math}\lim_{n\rightarrow \infty}x=0\end{math},
set here in in-line math style, looks slightly different when
set in display style.  (See next section).

\subsubsection{Display Equations}
A numbered display equation -- one set off by vertical space
from the text and centered horizontally -- is produced
by the \textbf{equation} environment. An unnumbered display
equation is produced by the \textbf{displaymath} environment.

Again, in either environment, you can use any of the symbols
and structures available in \LaTeX; this section will just
give a couple of examples of display equations in context.
First, consider the equation, shown as an inline equation above:
\begin{equation}\lim_{n\rightarrow \infty}x=0\end{equation}
Notice how it is formatted somewhat differently in
the \textbf{displaymath}
environment.  Now, we'll enter an unnumbered equation:
\begin{displaymath}\sum_{i=0}^{\infty} x + 1\end{displaymath}
and follow it with another numbered equation:
\begin{equation}\sum_{i=0}^{\infty}x_i=\int_{0}^{\pi+2} f\end{equation}
just to demonstrate \LaTeX's able handling of numbering.

\subsection{Citations}
Citations to articles \cite{bowman:reasoning,
clark:pct, braams:babel, herlihy:methodology},
conference proceedings \cite{clark:pct} or
books \cite{salas:calculus, Lamport:LaTeX} listed
in the Bibliography section of your
article will occur throughout the text of your article.
You should use BibTeX to automatically produce this bibliography;
you simply need to insert one of several citation commands with
a key of the item cited in the proper location in
the \texttt{.tex} file \cite{Lamport:LaTeX}.
The key is a short reference you invent to uniquely
identify each work; in this sample document, the key is
the first author's surname and a
word from the title.  This identifying key is included
with each item in the \texttt{.bib} file for your article.

The details of the construction of the \texttt{.bib} file
are beyond the scope of this sample document, but more
information can be found in the \textit{Author's Guide},
and exhaustive details in the \textit{\LaTeX\ User's
Guide} \cite{Lamport:LaTeX} and in various online source.\footnote{See e.g. \url{http://www.bibtex.org/}.}

This article shows only the plainest form
of the citation command, using \texttt{{\char'134}cite}.
This is what is stipulated in the SIGS style specifications.
No other citation format is endorsed or supported.

\subsection{Blind Review}
A convenient flag has been provided that will allow you to easily mask any text
or reference that could be used to identify you as the author. To enable the
feature, comment the line \texttt{\textbackslash usepackage\{anonymize\}} and
uncomment the \texttt{\textbackslash usepackage[blind]\{anonymize\}} lines near
the top of the file. When \texttt{\textbackslash usepackage[blind]\{anonymize\}}
is enabled, anonymized text will be blackened out and masked bibliography names
and titles will be anonymised. For example, the following will cause the package
to generate a blind review version of the manuscript.\\

\texttt{\%\textbackslash usepackage\{anonymize\}}  \\
\texttt{\textbackslash usepackage[blind]\{anonymize\}} \\


The following configuration will cause the manuscript to disable masking text and produce the version of the manuscript for publication.

\texttt{\textbackslash usepackage\{anonymize\}} \\
\texttt{\%\textbackslash usepackage[blind]\{anonymize\}} \\

\subsubsection{Masking text}
To mask text that may identify you, place the text you want to hide inside the \textbackslash anonymize function. If you have enabled the blind review feature, \anonymize{you will not be able to read this text}. Change back to the non-blind mode when you are ready to publish. 

\subsubsection{Masking publications}
In order to mask your publications,  you will need to make a copy of your personal references, using the second version with a key name adding \texttt{-hidden}. For example, if your reference key is \texttt{myotherpublication}, make a copy called \texttt{myotherpublication-hidden} that has text displayed how you would like it displayed in blind review mode.  Wrap your reference name with the \textit{anoncite} function. For example, to make \texttt{{\char'134}cite\{mypublication\}} anonymous in the blind version, use the following instead.\\
\texttt{{\char'134}cite\{{\char'134}anoncite\{mypublication\}\}}.
You will note that \cite{\anoncite{mypublication}} and \cite{\anoncite{myotherpublication}} were written by me, so in blind mode, the identifying parts will be suppressed.

\subsection{Links}

Links to URLs can be included using the \texttt{{\char'134}url} command. This will generate a clickable link like this: \url{https://www.nime.org}.


\section{Conclusions}
This paragraph will end the body of this sample document.
Remember that you might still have Acknowledgments or
Appendices; brief samples of these
follow.  There is still the Bibliography to deal with; and
we will make a disclaimer about that here: with the exception
of the reference to the \LaTeX\ book, the citations in
this paper are to articles which have nothing to
do with the present subject and are used as
examples only.
%\end{document}  % This is where a 'short' article might terminate

%ACKNOWLEDGMENTS are optional
\section{Acknowledgments}

This section is optional and a place for you to acknowledge non-author contributors, grants or funding, or any other support received which you would like to recognise. In general, this section should be anonymised in initial submissions.


\section{Ethical Standards}
To ensure objectivity and transparency in research and to ensure that accepted principles of ethical and professional conduct have been followed, authors must include a section “Ethical Standards” before the References. 
This section should include (if relevant): information regarding sources of funding, potential conflicts of interest (financial or non-financial), informed consent if the research involved human participants, statement on welfare of animals if the research involved animals or any other information or context that helps ethically situate your research.
For help with the ethics section, feel free to ask on the NIME forum: \url{https://forum.nime.org}.

\bibliography{references}

%%% Place this command where you want to balance the columns on the last page. 
%\balancecolumns 

\end{document}
